\input{./src/main.sty}
% Additional SI unit for Fahrenheit
\DeclareSIUnit\fahrenheit{\degree F}

\begin{document}

% Include title page
\input{./src/titlepage.tex}

\pagebreak

\begin{enumerate}
    \item 
        \begin{enumerate}[a.]
            \item (5 points) How many possible microstates exist for a system containing eight atoms of element
            A and one atom of element B? Draw all possible microstates for such a system on a three-by-three
            two-dimensional lattice.
            \item (5 points) How many possible microstates exist for a system containing four atoms of element A,
            two atoms of element B, and three atoms of element C?
        \item (5 points) The following lattices show microstates of a 4x4 lattice. Which is the most probable
            microstate consistent with the macrostate of being \SI{50}{\percent} occupied. Explain your answer.
        \end{enumerate}

    \begin{figure}[h]
        \centering
        \includegraphics[width=0.6\textwidth]{./assets/fig_1.png}
    \end{figure}

    \pagebreak

    \item Consider a small colloidal glass bead of mass m in a container of water held at temperature $T$. The
    gravitational potential energy $U$ of the glass particle depends on its height above the bottom of the
    container z: $U(z) = mgz$, where $g$ is the acceleration due to gravity.

    \begin{enumerate}
        \item Determine the probability $P(z)$ that the bead is found at a certain height $z$. Normalize the
            probability distribution such that $\int_0^\infty P(z) = 1$.
        \item Calculate the average height of the bead $\langle z\rangle$. Under what conditions would we expect particles
            to “sediment” out of a solution?
        \item In the high temperature limit (or $T \rightarrow \infty$), where can we expect to find the bead? \label{c}
        \item Calculate the variance of the bead height $\langle z^2\rangle - \langle z\rangle^2$. \label{d}
        \item Given what we have learned in part \ref{d}, re-interpret your answer to part \ref{c}. That is, at any
            instant in time, where might the bead be found at high temperatures?
        \item Calculate the average energy $\langle U \rangle$.
    \end{enumerate}
    
    \pagebreak

    \item One simple model of a crystal consists of a collection of masses and springs. Consider one particular
        “normal mode” of the crystal, characterized by natural frequency $\omega$. The vibrational energy associated
        with this normal mode can be quantized such that the $n$\textsuperscript{th} energy level is given by

    \begin{equation*}
        \epsilon_n = \left(n + \frac{1}{2}\right)\hbar \omega
    \end{equation*}


    where $n = 0, 1, 2, \dots$ and $\hbar$ is Planck’s constant divided by $2\pi$. This is called the “Einstein model”
    of a crystalline solid.

    \begin{figure}[h]
        \centering
        \includegraphics[width=0.4\textwidth]{./assets/fig_2.png}
    \end{figure}
    
    \begin{enumerate}[(a)]

        \item Calculate the partition function $Z$. To simplify, take advantage of the formula for a convergent
            geometric series:

        \begin{equation*}
            \sum_{n=0}^\infty x^n = \frac{1}{1 - x}
        \end{equation*}

        \item Calculate the entropy, internal energy, and Helmholtz free energy of the crystal. 
            You can express your results in terms of the “Einstein temperature” $\vartheta_E  ≡ \hbar \omega/k_B$. 
            (Note that for diamonds, $\vartheta_E \approx \SI{1320}{K}$.)

        \item Calculate the molar heat capacity $C_V$ . Plot $C_V /k_B$ as a function of $T /\vartheta_E$. (This model actually
            does a good job at capturing the high-temperature behavior of real crystals! The low-temperature
            requires some corrections, however.)
    \end{enumerate}




    
\end{enumerate}

% \section*{Supporting code:}
% \inputminted{julia}{./calculations/src/calculations.jl}

\end{document}

% Additional SI unit for Fahrenheit
\DeclareSIUnit\fahrenheit{\degree F}

\begin{document}

% Include title page
\input{./src/titlepage.tex}

\pagebreak

\begin{enumerate}
    \item 
        \begin{enumerate}[a.]
            \item (5 points) How many possible microstates exist for a system containing eight atoms of element
            A and one atom of element B? Draw all possible microstates for such a system on a three-by-three
            two-dimensional lattice.
            \item (5 points) How many possible microstates exist for a system containing four atoms of element A,
            two atoms of element B, and three atoms of element C?
        \item (5 points) The following lattices show microstates of a 4x4 lattice. Which is the most probable
            microstate consistent with the macrostate of being \SI{50}{\percent} occupied. Explain your answer.
        \end{enumerate}

    \begin{figure}[h]
        \centering
        \includegraphics[width=0.6\textwidth]{./assets/fig_1.png}
    \end{figure}

    \pagebreak

    \item Consider a small colloidal glass bead of mass m in a container of water held at temperature $T$. The
    gravitational potential energy $U$ of the glass particle depends on its height above the bottom of the
    container z: $U(z) = mgz$, where $g$ is the acceleration due to gravity.

    \begin{enumerate}
        \item Determine the probability $P(z)$ that the bead is found at a certain height $z$. Normalize the
            probability distribution such that $\int_0^\infty P(z) = 1$.
        \item Calculate the average height of the bead $\langle z\rangle$. Under what conditions would we expect particles
            to “sediment” out of a solution?
        \item In the high temperature limit (or $T \rightarrow \infty$), where can we expect to find the bead? \label{c}
        \item Calculate the variance of the bead height $\langle z^2\rangle - \langle z\rangle^2$. \label{d}
        \item Given what we have learned in part \ref{d}, re-interpret your answer to part \ref{c}. That is, at any
            instant in time, where might the bead be found at high temperatures?
        \item Calculate the average energy $\langle U \rangle$.
    \end{enumerate}
    
    \pagebreak

    \item One simple model of a crystal consists of a collection of masses and springs. Consider one particular
        “normal mode” of the crystal, characterized by natural frequency $\omega$. The vibrational energy associated
        with this normal mode can be quantized such that the $n$\textsuperscript{th} energy level is given by

    \begin{equation*}
        \epsilon_n = \left(n + \frac{1}{2}\right)\hbar \omega
    \end{equation*}


    where $n = 0, 1, 2, \dots$ and $\hbar$ is Planck’s constant divided by $2\pi$. This is called the “Einstein model”
    of a crystalline solid.

    \begin{figure}[h]
        \centering
        \includegraphics[width=0.4\textwidth]{./assets/fig_2.png}
    \end{figure}
    
    \begin{enumerate}[(a)]

        \item Calculate the partition function $Z$. To simplify, take advantage of the formula for a convergent
            geometric series:

        \begin{equation*}
            \sum_{n=0}^\infty x^n = \frac{1}{1 - x}
        \end{equation*}

        \item Calculate the entropy, internal energy, and Helmholtz free energy of the crystal. 
            You can express your results in terms of the “Einstein temperature” $\vartheta_E  ≡ \hbar \omega/k_B$. 
            (Note that for diamonds, $\vartheta_E \approx \SI{1320}{K}$.)

        \item Calculate the molar heat capacity $C_V$ . Plot $C_V /k_B$ as a function of $T /\vartheta_E$. (This model actually
            does a good job at capturing the high-temperature behavior of real crystals! The low-temperature
            requires some corrections, however.)
    \end{enumerate}




    
\end{enumerate}

% \section*{Supporting code:}
% \inputminted{julia}{./calculations/src/calculations.jl}

\end{document}

% Additional SI unit for Fahrenheit
\DeclareSIUnit\fahrenheit{\degree F}

\begin{document}

% Include title page
\input{./src/titlepage.tex}

\pagebreak

\begin{enumerate}
    \item 
        \begin{enumerate}[a.]
            \item (5 points) How many possible microstates exist for a system containing eight atoms of element
            A and one atom of element B? Draw all possible microstates for such a system on a three-by-three
            two-dimensional lattice.
            \item (5 points) How many possible microstates exist for a system containing four atoms of element A,
            two atoms of element B, and three atoms of element C?
        \item (5 points) The following lattices show microstates of a 4x4 lattice. Which is the most probable
            microstate consistent with the macrostate of being \SI{50}{\percent} occupied. Explain your answer.
        \end{enumerate}

    \begin{figure}[h]
        \centering
        \includegraphics[width=0.6\textwidth]{./assets/fig_1.png}
    \end{figure}

    \pagebreak

    \item Consider a small colloidal glass bead of mass m in a container of water held at temperature $T$. The
    gravitational potential energy $U$ of the glass particle depends on its height above the bottom of the
    container z: $U(z) = mgz$, where $g$ is the acceleration due to gravity.

    \begin{enumerate}
        \item Determine the probability $P(z)$ that the bead is found at a certain height $z$. Normalize the
            probability distribution such that $\int_0^\infty P(z) = 1$.
        \item Calculate the average height of the bead $\langle z\rangle$. Under what conditions would we expect particles
            to “sediment” out of a solution?
        \item In the high temperature limit (or $T \rightarrow \infty$), where can we expect to find the bead? \label{c}
        \item Calculate the variance of the bead height $\langle z^2\rangle - \langle z\rangle^2$. \label{d}
        \item Given what we have learned in part \ref{d}, re-interpret your answer to part \ref{c}. That is, at any
            instant in time, where might the bead be found at high temperatures?
        \item Calculate the average energy $\langle U \rangle$.
    \end{enumerate}
    
    \pagebreak

    \item One simple model of a crystal consists of a collection of masses and springs. Consider one particular
        “normal mode” of the crystal, characterized by natural frequency $\omega$. The vibrational energy associated
        with this normal mode can be quantized such that the $n$\textsuperscript{th} energy level is given by

    \begin{equation*}
        \epsilon_n = \left(n + \frac{1}{2}\right)\hbar \omega
    \end{equation*}


    where $n = 0, 1, 2, \dots$ and $\hbar$ is Planck’s constant divided by $2\pi$. This is called the “Einstein model”
    of a crystalline solid.

    \begin{figure}[h]
        \centering
        \includegraphics[width=0.4\textwidth]{./assets/fig_2.png}
    \end{figure}
    
    \begin{enumerate}[(a)]

        \item Calculate the partition function $Z$. To simplify, take advantage of the formula for a convergent
            geometric series:

        \begin{equation*}
            \sum_{n=0}^\infty x^n = \frac{1}{1 - x}
        \end{equation*}

        \item Calculate the entropy, internal energy, and Helmholtz free energy of the crystal. 
            You can express your results in terms of the “Einstein temperature” $\vartheta_E  ≡ \hbar \omega/k_B$. 
            (Note that for diamonds, $\vartheta_E \approx \SI{1320}{K}$.)

        \item Calculate the molar heat capacity $C_V$ . Plot $C_V /k_B$ as a function of $T /\vartheta_E$. (This model actually
            does a good job at capturing the high-temperature behavior of real crystals! The low-temperature
            requires some corrections, however.)
    \end{enumerate}




    
\end{enumerate}

% \section*{Supporting code:}
% \inputminted{julia}{./calculations/src/calculations.jl}

\end{document}

% Additional SI unit for Fahrenheit
\DeclareSIUnit\fahrenheit{\degree F}

\begin{document}

% Include title page
\input{./src/titlepage.tex}

\pagebreak

\begin{enumerate}
    \item 
        \begin{enumerate}[a.]
            \item (5 points) How many possible microstates exist for a system containing eight atoms of element
            A and one atom of element B? Draw all possible microstates for such a system on a three-by-three
            two-dimensional lattice.
            \item (5 points) How many possible microstates exist for a system containing four atoms of element A,
            two atoms of element B, and three atoms of element C?
        \item (5 points) The following lattices show microstates of a 4x4 lattice. Which is the most probable
            microstate consistent with the macrostate of being \SI{50}{\percent} occupied. Explain your answer.
        \end{enumerate}

    \begin{figure}[h]
        \centering
        \includegraphics[width=0.6\textwidth]{./assets/fig_1.png}
    \end{figure}

    \pagebreak

    \item Consider a small colloidal glass bead of mass m in a container of water held at temperature $T$. The
    gravitational potential energy $U$ of the glass particle depends on its height above the bottom of the
    container z: $U(z) = mgz$, where $g$ is the acceleration due to gravity.

    \begin{enumerate}
        \item Determine the probability $P(z)$ that the bead is found at a certain height $z$. Normalize the
            probability distribution such that $\int_0^\infty P(z) = 1$.
        \item Calculate the average height of the bead $\langle z\rangle$. Under what conditions would we expect particles
            to “sediment” out of a solution?
        \item In the high temperature limit (or $T \rightarrow \infty$), where can we expect to find the bead? \label{c}
        \item Calculate the variance of the bead height $\langle z^2\rangle - \langle z\rangle^2$. \label{d}
        \item Given what we have learned in part \ref{d}, re-interpret your answer to part \ref{c}. That is, at any
            instant in time, where might the bead be found at high temperatures?
        \item Calculate the average energy $\langle U \rangle$.
    \end{enumerate}
    
    \pagebreak

    \item One simple model of a crystal consists of a collection of masses and springs. Consider one particular
        “normal mode” of the crystal, characterized by natural frequency $\omega$. The vibrational energy associated
        with this normal mode can be quantized such that the $n$\textsuperscript{th} energy level is given by

    \begin{equation*}
        \epsilon_n = \left(n + \frac{1}{2}\right)\hbar \omega
    \end{equation*}


    where $n = 0, 1, 2, \dots$ and $\hbar$ is Planck’s constant divided by $2\pi$. This is called the “Einstein model”
    of a crystalline solid.

    \begin{figure}[h]
        \centering
        \includegraphics[width=0.4\textwidth]{./assets/fig_2.png}
    \end{figure}
    
    \begin{enumerate}[(a)]

        \item Calculate the partition function $Z$. To simplify, take advantage of the formula for a convergent
            geometric series:

        \begin{equation*}
            \sum_{n=0}^\infty x^n = \frac{1}{1 - x}
        \end{equation*}

        \item Calculate the entropy, internal energy, and Helmholtz free energy of the crystal. 
            You can express your results in terms of the “Einstein temperature” $\vartheta_E  ≡ \hbar \omega/k_B$. 
            (Note that for diamonds, $\vartheta_E \approx \SI{1320}{K}$.)

        \item Calculate the molar heat capacity $C_V$ . Plot $C_V /k_B$ as a function of $T /\vartheta_E$. (This model actually
            does a good job at capturing the high-temperature behavior of real crystals! The low-temperature
            requires some corrections, however.)
    \end{enumerate}




    
\end{enumerate}

% \section*{Supporting code:}
% \inputminted{julia}{./calculations/src/calculations.jl}

\end{document}

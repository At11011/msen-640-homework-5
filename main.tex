\input{./src/main.sty}
% Additional SI unit for Fahrenheit
\DeclareSIUnit\fahrenheit{\degree F}

\begin{document}

% Include title page
\input{./src/titlepage.tex}

\pagebreak

\begin{enumerate}
    \item 
        \begin{enumerate}[a.]
            \item (5 points) How many possible microstates exist for a system containing eight atoms of element
            A and one atom of element B? Draw all possible microstates for such a system on a three-by-three
            two-dimensional lattice.
            \item (5 points) How many possible microstates exist for a system containing four atoms of element A,
            two atoms of element B, and three atoms of element C?
        \item (5 points) The following lattices show microstates of a 4x4 lattice. Which is the most probable
            microstate consistent with the macrostate of being \SI{50}{\percent} occupied. Explain your answer.
        \end{enumerate}

    \begin{figure}[h]
        \centering
        \includegraphics[width=0.6\textwidth]{./assets/fig_1.png}
    \end{figure}

    \pagebreak

    \item Consider a small colloidal glass bead of mass m in a container of water held at temperature $T$. The
    gravitational potential energy $U$ of the glass particle depends on its height above the bottom of the
    container z: $U(z) = mgz$, where $g$ is the acceleration due to gravity.

    \begin{enumerate}
        \item Determine the probability $P(z)$ that the bead is found at a certain height $z$. Normalize the
            probability distribution such that $\int_0^\infty P(z) = 1$.
        \item Calculate the average height of the bead $\langle z\rangle$. Under what conditions would we expect particles
            to “sediment” out of a solution?
        \item In the high temperature limit (or $T \rightarrow \infty$), where can we expect to find the bead? \label{c}
        \item Calculate the variance of the bead height $\langle z^2\rangle - \langle z\rangle^2$. \label{d}
        \item Given what we have learned in part \ref{d}, re-interpret your answer to part \ref{c}. That is, at any
            instant in time, where might the bead be found at high temperatures?
        \item Calculate the average energy $\langle U \rangle$.
    \end{enumerate}
    
    \pagebreak

    \item One simple model of a crystal consists of a collection of masses and springs. Consider one particular
        “normal mode” of the crystal, characterized by natural frequency $\omega$. The vibrational energy associated
        with this normal mode can be quantized such that the $n$\textsuperscript{th} energy level is given by

    \begin{equation*}
        \epsilon_n = \left(n + \frac{1}{2}\right)\hbar \omega
    \end{equation*}


    where $n = 0, 1, 2, \dots$ and $\hbar$ is Planck’s constant divided by $2\pi$. This is called the “Einstein model”
    of a crystalline solid.

    \begin{figure}[h]
        \centering
        \includegraphics[width=0.4\textwidth]{./assets/fig_2.png}
    \end{figure}
    
    \begin{enumerate}[(a)]

        \item Calculate the partition function $Z$. To simplify, take advantage of the formula for a convergent
            geometric series:

        \begin{equation*}
            \sum_{n=0}^\infty x^n = \frac{1}{1 - x}
        \end{equation*}

        \item Calculate the entropy, internal energy, and Helmholtz free energy of the crystal. 
            You can express your results in terms of the “Einstein temperature” $\vartheta_E  ≡ \hbar \omega/k_B$. 
            (Note that for diamonds, $\vartheta_E \approx \SI{1320}{K}$.)

        \item Calculate the molar heat capacity $C_V$ . Plot $C_V /k_B$ as a function of $T /\vartheta_E$. (This model actually
            does a good job at capturing the high-temperature behavior of real crystals! The low-temperature
            requires some corrections, however.)
    \end{enumerate}




    
\end{enumerate}

% \section*{Supporting code:}
% \inputminted{julia}{./calculations/src/calculations.jl}

\end{document}

% Additional SI unit for Fahrenheit
\DeclareSIUnit\fahrenheit{\degree F}

\begin{document}

% Include title page
\input{./src/titlepage.tex}

\pagebreak

\begin{enumerate}
    \item 
        \begin{enumerate}[a.]
            \item (5 points) How many possible microstates exist for a system containing eight atoms of element
            A and one atom of element B? Draw all possible microstates for such a system on a three-by-three
            two-dimensional lattice.
            \item (5 points) How many possible microstates exist for a system containing four atoms of element A,
            two atoms of element B, and three atoms of element C?
        \item (5 points) The following lattices show microstates of a 4x4 lattice. Which is the most probable
            microstate consistent with the macrostate of being \SI{50}{\percent} occupied. Explain your answer.
        \end{enumerate}

    \begin{figure}[h]
        \centering
        \includegraphics[width=0.6\textwidth]{./assets/fig_1.png}
    \end{figure}

    \pagebreak

    \item Consider a small colloidal glass bead of mass m in a container of water held at temperature $T$. The
    gravitational potential energy $U$ of the glass particle depends on its height above the bottom of the
    container z: $U(z) = mgz$, where $g$ is the acceleration due to gravity.

    \begin{enumerate}
        \item Determine the probability $P(z)$ that the bead is found at a certain height $z$. Normalize the
            probability distribution such that $\int_0^\infty P(z) = 1$.
        \item Calculate the average height of the bead $\langle z\rangle$. Under what conditions would we expect particles
            to “sediment” out of a solution?
        \item In the high temperature limit (or $T \rightarrow \infty$), where can we expect to find the bead? \label{c}
        \item Calculate the variance of the bead height $\langle z^2\rangle - \langle z\rangle^2$. \label{d}
        \item Given what we have learned in part \ref{d}, re-interpret your answer to part \ref{c}. That is, at any
            instant in time, where might the bead be found at high temperatures?
        \item Calculate the average energy $\langle U \rangle$.
    \end{enumerate}
    
    \pagebreak

    \item One simple model of a crystal consists of a collection of masses and springs. Consider one particular
        “normal mode” of the crystal, characterized by natural frequency $\omega$. The vibrational energy associated
        with this normal mode can be quantized such that the $n$\textsuperscript{th} energy level is given by

    \begin{equation*}
        \epsilon_n = \left(n + \frac{1}{2}\right)\hbar \omega
    \end{equation*}


    where $n = 0, 1, 2, \dots$ and $\hbar$ is Planck’s constant divided by $2\pi$. This is called the “Einstein model”
    of a crystalline solid.

    \begin{figure}[h]
        \centering
        \includegraphics[width=0.4\textwidth]{./assets/fig_2.png}
    \end{figure}
    
    \begin{enumerate}[(a)]

        \item Calculate the partition function $Z$. To simplify, take advantage of the formula for a convergent
            geometric series:

        \begin{equation*}
            \sum_{n=0}^\infty x^n = \frac{1}{1 - x}
        \end{equation*}

        \item Calculate the entropy, internal energy, and Helmholtz free energy of the crystal. 
            You can express your results in terms of the “Einstein temperature” $\vartheta_E  ≡ \hbar \omega/k_B$. 
            (Note that for diamonds, $\vartheta_E \approx \SI{1320}{K}$.)

        \item Calculate the molar heat capacity $C_V$ . Plot $C_V /k_B$ as a function of $T /\vartheta_E$. (This model actually
            does a good job at capturing the high-temperature behavior of real crystals! The low-temperature
            requires some corrections, however.)
    \end{enumerate}




    
\end{enumerate}

% \section*{Supporting code:}
% \inputminted{julia}{./calculations/src/calculations.jl}

\end{document}

% Additional SI unit for Fahrenheit
\DeclareSIUnit\fahrenheit{\degree F}

\begin{document}

% Include title page
\input{./src/titlepage.tex}

\pagebreak

\begin{enumerate}
    \item 
        \begin{enumerate}[a.]
            \item (5 points) How many possible microstates exist for a system containing eight atoms of element
            A and one atom of element B? Draw all possible microstates for such a system on a three-by-three
            two-dimensional lattice.
            \item (5 points) How many possible microstates exist for a system containing four atoms of element A,
            two atoms of element B, and three atoms of element C?
        \item (5 points) The following lattices show microstates of a 4x4 lattice. Which is the most probable
            microstate consistent with the macrostate of being \SI{50}{\percent} occupied. Explain your answer.
        \end{enumerate}

    \begin{figure}[h]
        \centering
        \includegraphics[width=0.6\textwidth]{./assets/fig_1.png}
    \end{figure}

    \pagebreak

    \item Consider a small colloidal glass bead of mass m in a container of water held at temperature $T$. The
    gravitational potential energy $U$ of the glass particle depends on its height above the bottom of the
    container z: $U(z) = mgz$, where $g$ is the acceleration due to gravity.

    \begin{enumerate}
        \item Determine the probability $P(z)$ that the bead is found at a certain height $z$. Normalize the
            probability distribution such that $\int_0^\infty P(z) = 1$.
        \item Calculate the average height of the bead $\langle z\rangle$. Under what conditions would we expect particles
            to “sediment” out of a solution?
        \item In the high temperature limit (or $T \rightarrow \infty$), where can we expect to find the bead? \label{c}
        \item Calculate the variance of the bead height $\langle z^2\rangle - \langle z\rangle^2$. \label{d}
        \item Given what we have learned in part \ref{d}, re-interpret your answer to part \ref{c}. That is, at any
            instant in time, where might the bead be found at high temperatures?
        \item Calculate the average energy $\langle U \rangle$.
    \end{enumerate}
    
    \pagebreak

    \item One simple model of a crystal consists of a collection of masses and springs. Consider one particular
        “normal mode” of the crystal, characterized by natural frequency $\omega$. The vibrational energy associated
        with this normal mode can be quantized such that the $n$\textsuperscript{th} energy level is given by

    \begin{equation*}
        \epsilon_n = \left(n + \frac{1}{2}\right)\hbar \omega
    \end{equation*}


    where $n = 0, 1, 2, \dots$ and $\hbar$ is Planck’s constant divided by $2\pi$. This is called the “Einstein model”
    of a crystalline solid.

    \begin{figure}[h]
        \centering
        \includegraphics[width=0.4\textwidth]{./assets/fig_2.png}
    \end{figure}
    
    \begin{enumerate}[(a)]

        \item Calculate the partition function $Z$. To simplify, take advantage of the formula for a convergent
            geometric series:

        \begin{equation*}
            \sum_{n=0}^\infty x^n = \frac{1}{1 - x}
        \end{equation*}

        \item Calculate the entropy, internal energy, and Helmholtz free energy of the crystal. 
            You can express your results in terms of the “Einstein temperature” $\vartheta_E  ≡ \hbar \omega/k_B$. 
            (Note that for diamonds, $\vartheta_E \approx \SI{1320}{K}$.)

        \item Calculate the molar heat capacity $C_V$ . Plot $C_V /k_B$ as a function of $T /\vartheta_E$. (This model actually
            does a good job at capturing the high-temperature behavior of real crystals! The low-temperature
            requires some corrections, however.)
    \end{enumerate}




    
\end{enumerate}

% \section*{Supporting code:}
% \inputminted{julia}{./calculations/src/calculations.jl}

\end{document}

% Additional SI unit for Fahrenheit
\DeclareSIUnit\fahrenheit{\degree F}

\begin{document}

% Include title page
\input{./src/titlepage.tex}

\pagebreak

\begin{enumerate}
  \item
    \begin{enumerate}[a.]
      \item (5 points) How many possible microstates exist for a
        system containing eight atoms of element
        A and one atom of element B? Draw all possible microstates
        for such a system on a three-by-three
        two-dimensional lattice.

        \boxedanswer{
          This problem consists of 8 atoms and 9 possible spaces. As such,
          it can be expressed as:

          \begin{align*}
            C(n,r) &= \frac{n!}{r!(n-r)!} \\
            &
            \begin{aligned}
              n &= 9 \\
              n &= 8 \\
            \end{aligned} \\
            C(9,8) &= \frac{9!}{8!\overbrace{1!}^1} \\
            C(9,8) &= \frac{9\cancel{\cdot 8\cdot
            7\cdots}}{\cancel{8\cdot 7 \cdot 6\cdots}} \\
            \Aboxed{C(9,8) &= 9}
          \end{align*}

          \centering
          \includegraphics[width=0.4\textwidth]{./assets/q1_answer_a.png}

        }

      \item (5 points) How many possible microstates exist for a
        system containing four atoms of element A,
        two atoms of element B, and three atoms of element C?

        \boxedanswer{
          \begin{align*}
            \Omega &= \frac{N_0!}{n_A!\cdot n_B\cdot n_C!} \\
            N_0 &= n_A + n_B + n_C \\
            \begin{aligned}
              n_A! = 4 \\
              n_B! = 2 \\
              n_C! = 3 \\
            \end{aligned} \\
            N_0 &= 4 + 2 + 3 = 9 \\
            \Omega &= \frac{9!}{4!\cdot 2\cdot 3!} \\
            \Omega &=
            \frac{9\cdot8\cdot7\cancel{\cdot6}\cdot5\cancel{\cdot4\cdot3\cdot2}}{\cancel{4\cdot3\cdot2\cdot}2\cancel{\cdot
            3\cdot 2}} \\
            \Omega &= \frac{9\cdot8\cdot7\cdot5}{2} \\
            \Aboxed{\Omega &= 1260}
          \end{align*}
        }

        \pagebreak

      \item (5 points) The following lattices show microstates of a
        4x4 lattice. Which is the most probable
        microstate consistent with the macrostate of being
        \SI{50}{\percent} occupied. Explain your answer.

    \end{enumerate}

    \begin{figure}[ht]
      \centering
      \includegraphics[width=0.5\textwidth]{./assets/fig_1.png}
    \end{figure}
    \boxedanswer{
      If each box represents a single microstate, all three are equally likely.
      However, there are equivalent microstates that can look identical to
      these configurations. There are more equivalent microstates for the third
      option (iii) than the other two, as there is less clustering and a more
      even spread of atoms. The first option is representative of only a
      single microstate, while the second option may have a few microstates.
      But both options are visually lower entropy (less random, more clustered)
      than the third option. Therefore, the third option is most probable
      because there are more equivalent microstates that appear identical
      to it.
    }

    \pagebreak

  \item Consider a small colloidal glass bead of mass m in a
    container of water held at temperature $T$. The
    gravitational potential energy $U$ of the glass particle depends
    on its height above the bottom of the
    container z: $U(z) = mgz$, where $g$ is the acceleration due to gravity.

    \begin{enumerate}
      \item Determine the probability $P(z)$ that the bead is found
        at a certain height $z$. Normalize the
        probability distribution such that $\int_0^\infty P(z) = 1$.

        \boxedanswer{
          Assume that the probability distribution follows a
          Boltzmann distribution:

          \begin{align*}
            P(z) &= Ae^{-\frac{U(z)}{k_BT}} \\
            &
            \begin{aligned}
              U(z) &= mgz \\
            \end{aligned} \\
            P(z) &= Ae^{-\frac{mgz}{k_BT}} \\
            \int_0^\infty Ae^{-\frac{mgz}{k_BT}} dz &= 1 \\
            A\frac{k_BT}{mg}\left[e^{-\frac{mgz}{k_BT}}\right]_0^\infty &= 1 \\
            A &= \frac{mg}{k_BT} \\
            \Aboxed{P(z) &= \frac{mg}{k_BT}e^{-\frac{mgz}{k_BT}}}
          \end{align*}
        }

      \item Calculate the average height of the bead $\langle
        z\rangle$. Under what conditions would we expect particles
        to “sediment” out of a solution?

        \boxedanswer{
          The average height can be found as follows:

          \begin{align*}
            \langle z\rangle &= \int_0^\infty
            z\frac{mg}{k_BT}e^{-\frac{mgz}{k_BT}}dz \\
            \langle z\rangle &= \frac{mg}{k_BT}\int_0^\infty
            ze^{-\frac{mgz}{k_BT}}dz \\
            &
            \begin{aligned}
              u &= \frac{mgz}{k_BT} \\
              du &= \frac{mg}{k_BT}dz \\
              dz &= \frac{k_BT}{mg}du \\
              z &= \frac{uk_BT}{mg} \\
            \end{aligned} \\
            \langle z\rangle &= \frac{mg}{k_BT}\int_0^\infty
            ze^{-\frac{mgz}{k_BT}}dz \\
            \langle z\rangle &= \frac{mg}{k_BT}\int_0^\infty
            \left( \frac{uk_BT}{mg} \right)e^{-u}\frac{k_BT}{mg}du \\
            \langle z\rangle &= \frac{k_BT}{mg}\int_0^\infty
            ue^{-u}du \\
            \langle z\rangle &=
            \frac{k_BT}{mg}\left[-\cancel{\left[ue^{-u}\right]_0^\infty} -
            \overbrace{\int_0^\infty e^{-u} du}^{-1}\right] \\
            \Aboxed{\langle z\rangle &= \frac{k_BT}{mg}}
          \end{align*}
        }

        \pagebreak

      \item In the high temperature limit (or $T \rightarrow
        \infty$), where can we expect to find the bead? \label{c}

        \boxedanswer{
          The Boltzmann distribution becomes very flat at high temperatures,
          and the average height scales with temperature. This indicates that
          the bead can be found anywhere, and is independent of
          height. Based on the
          expression for the "average" value, one may think that beads would
          be expected to "rise" with temperature, but this isn't
          necessarily true.
        }

      \item Calculate the variance of the bead height $\langle
        z^2\rangle - \langle z\rangle^2$. \label{d}

        \boxedanswer{
          \begin{align*}
            \langle z^2\rangle &= \int_0^\infty
            z^2\frac{mg}{k_BT}e^{-\frac{mgz}{k_BT}}dz \\
            & u = \frac{mgz}{k_BT} \\
            & z = \frac{uk_BT}{mg} \\
            & dz = \frac{k_BT}{mg}du \\
            \langle z^2\rangle &= \frac{mg}{k_BT}\int_0^\infty
            \left(\frac{uk_BT}{mg}\right)^2e^{-u}\frac{k_BT}{mg}du \\
            \langle z^2\rangle &= \left(\frac{k_BT}{mg}\right)^2\int_0^\infty
            u^2e^{-u}du \\
            \langle z^2\rangle &=
            -\left(\frac{k_BT}{mg}\right)^2\left[(u^2 + 2u
            +2)e^{-u}\right]_0^\infty \\
            \langle z^2\rangle &= 2\left(\frac{k_BT}{mg}\right)^2 \\
            \langle z \rangle &= \left(\frac{k_BT}{mg}\right) \\
            \Aboxed{\langle z^2 \rangle - \langle z \rangle^2 &=
            \left(\frac{k_BT}{mg}\right)^2}
          \end{align*}
        }

      \item Given what we have learned in part d, re-interpret
        your answer to part c. That is, at any
        instant in time, where might the bead be found at high temperatures?

        \boxedanswer{
          This clearly shows that the deviation grows faster than the mean,
          which means that the distribution flattens out with increasing
          temperature, and the liklihood of finding the bead at any point
          becomes approximately uniform.
        }

      \item Calculate the average energy $\langle U \rangle$.

        \boxedanswer{
          \begin{align*}
            \langle U \rangle &= mg\langle z \rangle \\
            \langle U \rangle &= mg\frac{k_BT}{mg} \\
            \Aboxed{\langle U \rangle &= k_BT}
          \end{align*}
        }

    \end{enumerate}

    \pagebreak

  \item One simple model of a crystal consists of a collection of
    masses and springs. Consider one particular
    “normal mode” of the crystal, characterized by natural frequency
    $\omega$. The vibrational energy associated
    with this normal mode can be quantized such that the
    $n$\textsuperscript{th} energy level is given by

    \begin{equation*}
      \epsilon_n = \left(n + \frac{1}{2}\right)\hbar \omega
    \end{equation*}

    where $n = 0, 1, 2, \dots$ and $\hbar$ is Planck’s constant
    divided by $2\pi$. This is called the “Einstein model”
    of a crystalline solid.

    \begin{figure}[h]
      \centering
      \includegraphics[width=0.4\textwidth]{./assets/fig_2.png}
    \end{figure}

    \begin{enumerate}[(a)]

      \item Calculate the partition function $Z$. To simplify, take
        advantage of the formula for a convergent
        geometric series:

        \begin{equation*}
          \sum_{n=0}^\infty x^n = \frac{1}{1 - x}
        \end{equation*}

        \boxedanswer{
            \begin{align*}
                Z &= \sum_{n=0}^\infty e^{-\frac{\epsilon_n}{k_BT}} \\
                  & \begin{aligned}
                    \epsilon_n &= \left(n + \frac{1}{2}\right)\hbar \omega \\
                  \end{aligned} \\
                Z &= \sum_{n=0}^\infty e^{-\frac{\hbar\omega}{k_BT}\left(n + \frac{1}{2}\right)} \\
                Z &= e^{-\frac{\hbar\omega}{2k_BT}}\sum_{n=0}^\infty \left(e^{-\frac{\hbar\omega}{k_BT}}\right)^n \\
                \Aboxed{Z &= \frac{e^{-\frac{\hbar\omega}{2k_BT}}}{1 - e^{-\frac{\hbar\omega}{k_BT}}}}
            \end{align*}
        }

        \pagebreak

      \item Calculate the entropy, internal energy, and Helmholtz
        free energy of the crystal.
        You can express your results in terms of the “Einstein
        temperature” $\vartheta_E  ≡ \hbar \omega/k_B$.
        (Note that for diamonds, $\vartheta_E \approx \SI{1320}{K}$.)

        \boxedanswer{
            \begin{align*}
                S &= N_0 k_B \ln Z + N_0 kT\left( \frac{\delta \ln Z}{\delta T} \right)_V \\
                  &\begin{aligned}
                    Z &= \frac{e^{-\frac{\hbar\omega}{2k_BT}}}{1 - e^{-\frac{\hbar\omega}{k_BT}}} \\
                    \ln Z &= -\frac{\hbar\omega}{2k_BT} - \ln\left(1 - e^{-\frac{\hbar\omega}{k_BT}}\right) \\
                      \frac{\delta \ln Z}{\delta T} &= 
                      \frac{\hbar w}{k_BT^2\left(e^{\frac{\hbar\omega}{k_BT}} - 1\right)}
                      + \frac{\hbar\omega}{2k_BT^2} \\
                \end{aligned} \\
                S &= N_0k_B\left[\ln\left( e^{-\frac{\hbar\omega}{2k_BT}}\right) 
                    - \ln\left(1 - e^{-\frac{\hbar\omega}{k_BT}}\right) \right] 
                    + N_0 k_BT\left( \frac{\hbar w}{k_BT^2\left(e^{\frac{\hbar\omega}{k_BT}} - 1\right)}
                      + \frac{\hbar\omega}{2k_BT^2} \right) \\
                \Aboxed{S &= \frac{N_0\hbar \omega}{T\left(e^{\frac{\vartheta_E}{T}} - 1\right)}
                -N_0k_B\ln\left(1 - e^{-\frac{\vartheta_E}{T}}\right)} \\
                    U &= N_0k_BT^2\left(\frac{\delta \ln Z}{\delta T}\right)_V \\
                    U &= N_0k_BT^2\left(\frac{\hbar w}{k_BT^2\left(e^{\frac{\hbar\omega}{k_BT}} - 1\right)}
                      + \frac{\hbar\omega}{2k_BT^2}\right) \\
                    U &= N_0\hbar\omega\left(\frac{1}{\left(e^{\frac{\hbar\omega}{k_BT}} - 1\right)}
                      + \frac{1}{2}\right) \\
                    \Aboxed{U &= \frac{N_0\hbar\omega}{2}\frac{\left(
                        1 + e^{-\frac{\vartheta}{T}}\right)}{
                    \left(1 - e^{-\frac{\vartheta}{T}}\right)}} \\
                    F &= -N_0k_BT\ln Z \\
                    F &= -N_0k_BT\left[-\frac{\hbar\omega}{2k_BT} - \ln\left(1 - e^{-\frac{\hbar\omega}{k_BT}}\right)\right] \\
                    \Aboxed{F &= \frac{N_0\hbar\omega}{2} + N_0k_BT\ln\left(1 - e^{-\frac{\vartheta_E}{T}}\right)}
            \end{align*}
        }

        \pagebreak

      \item Calculate the molar heat capacity $C_V$ . Plot $C_V /k_B$
        as a function of $T /\vartheta_E$. (This model actually
          does a good job at capturing the high-temperature behavior
          of real crystals! The low-temperature
        requires some corrections, however.)

        \boxedanswer{
            \begin{align*}
                C_V &= \left(\frac{\delta U}{\delta T}\right)_V \\
                C_V &= \frac{\delta}{\delta T}\left(
                    \frac{N_0\hbar\omega}{2}\frac{\left(
                        1 + e^{-\frac{\vartheta}{T}}\right)}{
                        \left(1 - e^{-\frac{\vartheta}{T}}\right)}
                \right) \\
                        \Aboxed{C_V &= \frac{N_0\hbar\omega\vartheta_E e^{\frac{\vartheta_E}{T}}}
                        {T^2\left(e^{\frac{\vartheta_E}{T}} - 1\right)^2}} \\
                            \frac{C_V}{k_B} &= N_0e^{\frac{\vartheta_E}{T}}\left(\frac{\vartheta_E}
                        {T\left(e^{\frac{\vartheta_E}{T}} - 1\right)}\right)^2 \\
            \end{align*}
        }

    \end{enumerate}

\end{enumerate}

\begin{figure}[h]
    \centering
        \includegraphics[width=0.6\textwidth]{./assets/Figure_1.png}
\end{figure}

% \section*{Supporting code:}
% \inputminted{julia}{./calculations/src/calculations.jl}

\end{document}
